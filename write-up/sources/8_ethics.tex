\subsection{Ethical Statement}

This dissertation project, using the XGBoost algorithm trained and tested on the CIC-IDS2017 dataset for NIDS, 
has been carried out in accordance with the University's ethical guidelines. The work has been reviewed and approved under 
the University's ethical framework for research. The project's general aim and specific task are to improve network safety and 
resilience to cyberattacks, which can be seen as beneficial to society, and one of the core ethical principles.

The project was granted ethical clearance on the \textbf{25-APR-25} and was placed into the \textbf{Risk Category 1}. This 
classification is due to the project's adherence to a research protocol that minimizes all potential risks. Specifically, 
the study utilizes the public and widely-used CIC-IDS2017 dataset, which consists of anonymized network traffic flows. No
personally identifiable information (PII) or real-time network data from private or institutional networks was collected, 
accessed, or analyzed during this dissertation project. This approach completely mitigates risks related to user privacy, 
data confidentiality, and potential misuse of information.

The main ethical implication of the NIDS technology is the theoretical potential to use it for surveillance purposes. At the same 
time, since the project's range is strictly limited to the academic and non-commercial environment, and its task is focused on 
detecting malicious user activity (e.g., DDoS attacks), rather than individual users themselves. The potential ethical risks 
associated with real-world deployment of the such NIDS system are not a primary concern of this dissertation. The project is 
transparent about its methods and purpose, contributing to the broader academic community's understanding of effective and 
ethical NIDS development.

The screenshot of the project's ethical clearance confirmation is attached in the appendix (see Figure\ref{fig:ethical_clearance}).