\begin{abstract}
  \noindent Network intrusion detection systems (NIDS) are vital to secure connected infrastructures, which have been expanding in recent years. While rule and signature-based approaches are able to detect known threats, they are not sufficient to tackle unknown or zero-day attacks, making effective machine learning-based NIDS an open problem. This dissertation proposes an end-to-end methodology for a realistic multi-class NIDS, with emphasis on the CIC-IDS2017 dataset, a highly imbalanced, large-scale collection of realistic network traffic data. A thorough data pre-processing pipeline has been developed to tackle real-world data challenges such as missing values and categorical features, with the addition of a missing-value generator. Most importantly, severe class-imbalance, which is characteristic of the multi-class NIDS problem, was addressed with the SMOTETomek over- and under-sampling technique. The model chosen for the multi-class classification problem is XGBoost, which was rigorously optimized through combined randomized and grid search. Experimental results show that the optimized model achieves an excellent Macro-averaged F1-score of 0.98, superior to other ML algorithms on this problem. The excellent performance achieved across rare classes in particular, verify the success of the proposed methodology, and also provides strong evidence of XGBoost as a highly applicable algorithm for multi-class NIDS. The analysis and rigorous testing of this pipeline is a detailed, evidence-based approach to the development of an automated intrusion detection system, and is a strong benchmark for future work.
\end{abstract}

