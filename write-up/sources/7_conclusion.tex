\subsection{Conclusion} 

In this dissertation, the effectiveness of the XGBoost algorithm for solving the NIDS problem on the CIC-IDS2017 dataset has been successfully investigated. A full end-to-end method for that task, which includes data pre-processing steps that are especially designed to address the most significant issues of this dataset and its machine learning, has been established. In addition, the process of training the XGBoost model on that data, hyperparameter tuning its configuration, and performing a detailed performance analysis of its predictions on the test dataset has been successfully completed, with its main results being presented and analyzed in the previous chapter.

In terms of a more detailed summary of this research and the main results of that work that have been established through its completion, it should be noted that, first of all, this work has confirmed the suitability of the chosen algorithm, XGBoost, for this task, as was hypothesized in Chapter 1, by demonstrating that it can achieve a level of performance, in terms of appropriate metrics like the per-class Precision, Recall, and F1-score, as high as or even higher than that of other state-of-the-art machine learning algorithms for this task. This conclusion, in turn, is based on the results of a rigorous comparative performance analysis and should be considered by other researchers and practitioners in the field of NIDS when selecting the most suitable algorithm for their needs.

On the other hand, this dissertation has also, on a more general level, demonstrated the high level of effectiveness of using sufficiently modern and powerful machine learning algorithms, especially those based on the well-established ensemble learning methods. This dissertation has also, in addition to the main results in terms of specific numbers and graphs, established a more detailed and comprehensive end-to-end pipeline for using those algorithms to solve the NIDS problem on a new and previously unstudied dataset.

\subsection{Future Work} 

The following future work is thus expected in this area:

\begin{itemize} 
\item \textbf{Other Datasets: } The model trained and tested on the CIC-IDS2017 dataset could be used to test its performance on other contemporary NIDS datasets, such as CSE-CIC-IDS2018, UNSW-NB15, or newer IoT/industrial control system datasets. This would provide a better understanding of the generality and robustness of the model across different network environments and attack types.
\item \textbf{Feature Engineering: } While this work has used a well-established feature extraction tool, additional features could be manually engineered or extracted using more advanced deep learning models, such as graph neural networks to capture more complex patterns or temporal dependencies in the network traffic.
\item \textbf{Hybrid Models: } A combination of XGBoost with deep learning approaches, where deep models act as feature extractors and XGBoost as the classifier, could be explored to leverage the strengths of both methodologies.
\item \textbf{Explainable AI (XAI): } Methods from XAI could be integrated to provide more insights into the decision-making process of the XGBoost model, potentially using techniques like SHAP or LIME to interpret individual predictions. This would be crucial for the model's adoption by cybersecurity professionals.
\item \textbf{Real-time Implementation: } The practical challenges of real-time data ingestion, processing, and decision-making latency in a live NIDS environment should be considered and investigated.
\item \textbf{Adversarial Attacks: } The model's robustness to adversarial attacks, specifically designed to evade detection, could be studied, and defense mechanisms could be developed and tested.
\item \textbf{Specific Attack Types: } Further studies could focus on the detection of specific attack categories that are more challenging or rare in the dataset, using specialized approaches or targeted data augmentation techniques.
\end{itemize} 